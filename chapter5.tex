\section{Chapter V.\hspace{0.2em} Irreducibility and	factorization in integral domains}
\subsection{\textsection1. Chain conditions and existence of factorizations}
Remember that in this section all rings are taken to be \emph{commutative}.

\hypertarget{Exercise V.1.1}{}
\begin{problem}[1.1]
	$\triangleright$ Let $R$ be a Noetherian ring, and let $I$ be an ideal of $R$. Prove that $R / I$ is a Noetherian ring. [\textsection1.1]
\end{problem}
\begin{solution}
	Let $\pi:R\to R / I$ be the projection. According to \hyperlink{Exercise III.4.2}{Exercise III.4.2}, the homomorphic image of the Noetherian ring $R$ is Noetherian. That is, $\pi(R)=R / I$ is Noetherian.
\end{solution}

\begin{problem}[1.2]
Prove that if $R[x]$ is Noetherian, so is $R$. (This is a `converse' to Hilbert's basis theorem.)
\end{problem}
\begin{solution}
According to \hyperlink{Exercise V.1.1}{Exercise V.1.1}, $R[x]$ is Noetherian implies $R\left[x\right]/\left(x\right)$ is Noetherian. Since in \hyperlink{Exercise III.4.12}{Exercise III.4.12} we have shown that
\[
R\cong R\left[x\right]/\left(x\right),
\]
we see $R$ is Noetherian.
\end{solution}

\subsection{\textsection4. Unique factorization in polynomial rings}
\hypertarget{Exercise V.4.7}{}
\begin{problem}[4.7]
$\triangleright$ A subset $S$ of a commutative ring $R$ is a \emph{multiplicative subset} (or \emph{multiplicatively closed}) if (i) $1 \in S$ and (ii) $s, t \in S \Longrightarrow s t \in S$. Define a relation on the set of pairs $(a, s)$ with $a \in R, s \in S$ as follows:
$$
(a, s) \sim\left(a^{\prime}, s^{\prime}\right) \Longleftrightarrow(\exists t \in S), t\left(s^{\prime} a-s a^{\prime}\right)=0 .
$$
Note that if $R$ is an integral domain and $S=R \backslash\{0\}$, then $S$ is a multiplicative subset, and the relation agrees with the relation introduced in $\S 4.2$.
\begin{itemize}
    \item Prove that the relation $\sim$ is an equivalence relation.
    \item Denote by $\frac{a}{s}$ the equivalence class of $(a, s)$, and define the same operations $+$, $\cdot$ on such `fractions' as the ones introduced in the special case of $\S 4.2$. Prove that these operations are well-defined.
    \item The set $S^{-1} R$ of fractions, endowed with the operations $+$, $\cdot$ is the \emph{localization} of $R$ at the multiplicative subset $S$. Prove that $S^{-1} R$ is a commutative ring and that the function $a \mapsto \frac{a}{1}$ defines a ring homomorphism $\ell: R \rightarrow S^{-1} R$.
    \item Prove that $\ell(s)$ is invertible for every $s \in S$.
    \item Prove that $R \rightarrow S^{-1} R$ is initial among ring homomorphisms $f: R \rightarrow R^{\prime}$ such that $f(s)$ is invertible in $R^{\prime}$ for every $s \in S$.
    \item Prove that $S^{-1} R$ is an integral domain if $R$ is an integral domain.
    \item Prove that $S^{-1} R$ is the zero-ring if and only if $0 \in S$.
\end{itemize}
$[4.8,4.9,4.11,4.15$, VII.2.16, VIII.1.4, VIII.2.5, VIII.2.6, VIII.2.12, §IX.9.1]
\end{problem}
\begin{solution}
    \begin{itemize}
        \item Reflexivity. 
        \begin{align*}
            1(sa-sa)=0\implies (a,s)\sim(a,s).
        \end{align*}
        Symmetry.
        \begin{align*}
            &(a, s) \sim\left(a^{\prime}, s^{\prime}\right) \Longleftrightarrow\left(\exists t \in S\right), t\left(s^{\prime} a-s a^{\prime}\right)=0 \\
            \iff&  (\exists t \in S), t\left(s a^{\prime}-s^{\prime} a\right)=0\iff \left(a^{\prime}, s^{\prime}\right)  \sim (a, s).
        \end{align*}
        Transitivity.
        \begin{align*}
            &(a, s) \sim\left(a^{\prime}, s^{\prime}\right) ,\left(a^{\prime}, s^{\prime}\right) \sim \left(a'', s''\right)\\
            \implies &\left(\exists t_1,t_2 \in S\right), t_1\left(s^{\prime} a-s a^{\prime}\right)=t_2\left(s'' a'-s' a''\right)=0  \\
            \implies & (\exists t_1t_2s' \in S), t_1t_2s'\left(s'' a-s a''\right)=(t_2s'')(t_1s'a)-(t_1s)(t_2s'a'')\\
            &\hspace{13.25em}=(t_2s'')(t_1sa')-(t_1s')(t_2s'' a')\\
            &\hspace{13.25em}=0\\
            \implies & (a, s)\sim \left(a'', s''\right) .
        \end{align*}
        \item Define
        \begin{align*}
            \frac{a_1}{s_1}+\frac{a_2}{s_2}&=\frac{a_1s_2+a_2s_1}{s_1s_2},\qquad \frac{a_1}{s_1}\cdot\frac{a_2}{s_2}=\frac{a_1a_2}{s_1s_2}.
        \end{align*}
        If $\frac{a_1}{s_1} = \frac{a_1^\prime}{s_1^\prime}$, then there exists $t\in S$ such that $ts_1^\prime a_1= ts_1 a_1^\prime$. Hence
        \begin{align*}
            &\;t(s_1^\prime s_2)(a_1s_2+a_2s_1)-t(s_1s_2)(a_1^\prime s_2+a_2s_1^\prime)\\
            =&\;t\left(s_1^\prime s_2a_1s_2+s_1^\prime s_2a_2s_1-s_1s_2a_1^\prime s_2-s_1s_2a_2s_1^\prime\right)\\
            =&\;t\left(s_1^\prime s_2a_1s_2+s_1^\prime s_2a_2s_1-s_1^\prime s_2a_1 s_2-s_1^\prime s_2a_2s_1\right)a\\
            =&\;0,
        \end{align*}
        which implies
        \begin{align*}
            \frac{a_1}{s_1}+\frac{a_2}{s_2}=\frac{a_1s_2+a_2s_1}{s_1s_2}= \frac{a_1^\prime s_2+a_2s_1^\prime}{s_1^\prime s_2}=\frac{a_1^\prime}{s_1^\prime}+\frac{a_2}{s_2}.
        \end{align*}
        If $\frac{a_2}{s_2} = \frac{a_2^\prime}{s_2^\prime}$, in a similar way, we can prove
        \begin{align*}
            \frac{a_1}{s_1}+\frac{a_2}{s_2}=\frac{a_1s_2+a_2s_1}{s_1s_2}= \frac{a_1s_2^\prime+a_2^\prime s_1}{s_1s_2^\prime}=\frac{a_1}{s_1}+\frac{a_2^\prime}{s_2^\prime}.
        \end{align*}
        Therefore, $+$ is well-defined.

        If $\frac{a_1}{s_1} = \frac{a_1^\prime}{s_1^\prime}$ and $\frac{a_2}{s_2} = \frac{a_2^\prime}{s_2^\prime}$, then there exists $t_1,t_2\in S$ such that $t_1s_1^\prime a_1= t_1s_1 a_1^\prime$ and $t_2s_2^\prime a_2= t_2s_2 a_2^\prime$. Hence we have
        \begin{align*}
           (t_1t_2)(s_1^\prime s_2^\prime a_1a_2 - s_1 s_2 a_1^\prime a_2^\prime)&=(t_1s_1^\prime a_1 )( t_2s_2^\prime a_2) -(t_1 s_1a_1^\prime )  (t_2s_2a_2^\prime )\\
           &=(t_1s_1 a_1^\prime )( t_2s_2 a_2^\prime) -(t_1 s_1a_1^\prime )  (t_2s_2a_2^\prime )\\
           &=0,
        \end{align*}
        which implies
        \begin{align*}
            \frac{a_1}{s_1}\cdot \frac{a_2}{s_2}=\frac{a_1a_2}{s_1s_2}= \frac{a_1^\prime a_2^\prime}{s_1^\prime s_2^\prime}=\frac{a_1^\prime}{s_1^\prime}\cdot \frac{a_2^\prime}{s_2^\prime}.
        \end{align*}
        Therefore, $\cdot$ is well-defined.
        \item It is straightforward to check that $S^{-1} R$ is a commutative ring. Define
        \begin{align*}
            \ell: R &\longrightarrow S^{-1} R\\
            a &\longmapsto \frac{a}{1}.
        \end{align*}
        Then
        \begin{align*}
            \ell(a+b)&=\frac{a+b}{1}=\frac{a}{1}+\frac{b}{1}= \ell(a)+\ell(b),\\
            \ell(ab)&=\frac{ab}{1}=\frac{a}{1}\cdot\frac{b}{1}=\ell(a)\ell(b).
        \end{align*}
        Hence $\ell$ is a ring homomorphism.
        \item Since
        \begin{align*}
            \frac{\ell(s)}{1}\frac{1}{s}=\frac{\ell(s)}{s}=\frac{s}{s}=\frac{1}{1},
        \end{align*}
        $\ell(s)$ is invertible in $S^{-1}R$ for every $s \in S$.
        \item 
        \[\xymatrix{
			S^{-1}R
            \ar@{-->}[r]^{g} & R'\\
			R\ar[ru]_{f}\ar[u]^{\ell} &    
		}\]
        Suppose $f: R \rightarrow R^{\prime}$ is a ring homomorphism such that $f(s)$ is invertible in $R^{\prime}$ for every $s \in S$. Define
        \begin{align*}
            g: S^{-1}R &\longrightarrow R^{\prime}\\
            \frac{a}{s} &\longmapsto f(s)^{-1}f(a).
        \end{align*}
        We can check that $g$ is a ring homomorphism as follows:
        \begin{align*}
           g\left(\frac{a_1}{s_1}+\frac{a_2}{s_2}\right)&=g\left(\frac{s_1a_2+s_2a_1}{s_1s_2}\right)\\
           &=f(s_1s_2)^{-1}f(s_1a_2+s_2a_1)\\
              &=f(s_1)^{-1}f(s_2)^{-1}f(s_1)f(a_2)+f(s_1)^{-1}f(s_2)^{-1}f(s_2)f(a_1)\\
           &=f(s_1)^{-1}f(a_1)+f(s_2)^{-1}f(a_2)\\
           &=g\left(\frac{a_1}{s_1}\right)+g\left(\frac{a_2}{s_2}\right),
        \end{align*}
        \begin{align*}
           g\left(\frac{a_1}{s_1}\cdot \frac{a_2}{s_2}\right)&=g\left(\frac{a_1a_2}{s_1s_2}\right)\\
              &=f(s_1s_2)^{-1}f(a_1a_2)\\
                &=f(s_1)^{-1}f(a_1)f(s_2)^{-1}f(a_2)\\
                &=g\left(\frac{a_1}{s_1}\right) g\left(\frac{a_2}{s_2}\right),
        \end{align*}
        \begin{align*}
            g\left(\frac{1}{1}\right)&=f(1)^{-1}f(1)=1.
        \end{align*}
        Since
    \begin{align*}
        g(\ell(a))&=g\left(\frac{a}{1}\right)=f(1)^{-1}f(a)=f(a),
    \end{align*}
    we see $g$ gives a commutative diagram. 
    
    Suppose $g'$ is another ring homomorphism such that $g'\circ \ell=f$. Then we have
    \begin{align*}
        g'\left(\frac{a}{s}\right)&=  g'\left(\frac{1}{s}\frac{a}{1}\right)\\
        &=g'\left(\frac{1}{s}\right)g'\left(\frac{a}{1}\right)\\
       &= \left(g'\left(\frac{s}{1}\right)\right)^{-1}g'\left(\frac{a}{1}\right)\\
        &=f(s)^{-1}f(a)\\
        &=g\left(\frac{a}{s}\right),
    \end{align*}
    which implies $g=g'$. Therefore, $\ell:R \rightarrow S^{-1} R$ is initial among ring homomorphisms $f: R \rightarrow R^{\prime}$ such that $f(s)$ is invertible in $R^{\prime}$ for every $s \in S$.
        \item If $R$ is an integral domain, then for any $\frac{a_1}{s_1}, \frac{a_2}{s_2} \in S^{-1} R$, we have
        \begin{align*}
            \frac{a_1}{s_1}\cdot\frac{a_2}{s_2}=\frac{a_1a_2}{s_1s_2}=\frac{0}{1}\implies a_1a_2=0\implies a_1=0\text{ or }a_2=0.
        \end{align*}
        That means $S^{-1} R$ is an integral domain.
        \item If $0 \in S$, then for any $a,a'\in R$, $s,s' \in S$, there exists $0\in S$ such that $0(s'a-sa')=0$. Therefore, $S^{-1} R$ only contains 1 element, which implies $S^{-1} R$ is a zero-ring.
    \end{itemize}
\end{solution}

\hypertarget{Exercise V.4.8}{}
\begin{problem}[4.8]
$\neg$ Let $S$ be a multiplicative subset of a commutative ring $R$, as in \hyperlink{Exercise V.4.7}{Exercise V.4.7}. For every $R$-module $M$, define a relation $\sim$ on the set of pairs $(m, s)$, where $m \in M$ and $s \in S$:
$$
(m, s) \sim\left(m^{\prime}, s^{\prime}\right) \Longleftrightarrow(\exists t \in S), t\left(s^{\prime} m-s m^{\prime}\right)=0 .
$$
Prove that this is an equivalence relation, and define an $S^{-1} R$-module structure on the set $S^{-1} M$ of equivalence classes, compatible with the $R$-module structure on $M$. The module $S^{-1} M$ is the localization of $M$ at $S$. [4.9, 4.11, 4.14, VIII.1.4, VIII.2.5, VIII.2.6]
\end{problem}
\begin{solution}
    We can check that $\sim$ is an equivalence relation as follows:\\
Reflexivity.
    \begin{align*}
         (\exists 1 \in S), 1\cdot\left(s m-s m\right)=0\implies (m, s) \sim\left(m, s\right).
    \end{align*}
    Symmetry.
    \begin{align*}
        &(m, s) \sim\left(m^{\prime}, s^{\prime}\right) \Longleftrightarrow\left(\exists t \in S\right), t\left(s^{\prime} m-s a^{\prime}\right)=0 \\
        \iff&  (\exists t \in S), t\left(sm^{\prime}-s^{\prime} m\right)=0\iff \left(m^{\prime}, s^{\prime}\right)  \sim (m, s).
    \end{align*}
    Transitivity.
    \begin{align*}
        &(m, s) \sim\left(m^{\prime}, s^{\prime}\right) ,\left(m^{\prime}, s^{\prime}\right) \sim \left(m'', s''\right)\\
        \implies &\left(\exists t_1,t_2 \in S\right), t_1\left(s^{\prime} m-s m^{\prime}\right)=t_2\left(s'' m'-s' m''\right)=0  \\
        \implies & (\exists t_1t_2s' \in S), t_1t_2s'\left(s'' m-s m''\right)=(t_2s'')(t_1s'm)-(t_1s)(t_2s'm'')\\
        &\hspace{13.25em}=(t_2s'')(t_1sm')-(t_1s')(t_2s'' m')\\
        &\hspace{13.25em}=0\\
        \implies & (m, s)\sim \left(m'', s''\right) .
    \end{align*}
    Denote by $\frac{m}{s}$ the equivalence class of $(m, s)$. The addtion of $S^{-1} M$ is defined as follows:
    \begin{align*}
        \frac{m_1}{s_1}+\frac{m_2}{s_2}&=\frac{s_2m_1+s_1m_2}{s_1s_2}.
    \end{align*}
    We can show that the addition is well-defined and makes $S^{-1} M$ an abelian group.\\
    Define the scalar multiplication of $S^{-1}R$ on $S^{-1} M$ as follows:
    \begin{align*}
        \frac{a}{s_1}\cdot\frac{m}{s_2}&=\frac{am}{s_1s_2}.
    \end{align*}
    We can show that the scalar multiplication is well-defined as follows:
    \begin{align*}
        &\frac{a}{s_1}=\frac{a'}{s_1'},\frac{m}{s_2}=\frac{m'}{s_2'}\\
        \implies& (\exists t_1,t_2 \in S), t_1\left(s_1a^{\prime}-s_1^{\prime} a\right)=0,t_2\left(s_2m^{\prime}-s_2^{\prime} m\right)=0\\
        \implies& (\exists t_1t_2 \in S), t_1t_2\left(s_1's_2'am-s_1s_2a^{\prime}m^{\prime}\right)=0\\
        \implies& \frac{am}{s_1s_2}=\frac{a'm'}{s_1's_2'}.
    \end{align*}
    We can check that the scalar multiplication is compatible with the addition as follows:
    \begin{align*}
        \frac{a}{s_1}\left(\frac{m_1}{s_2}+\frac{m_2}{s_3}\right)&= \frac{a}{s_1}\frac{s_3m_2+s_2m_1}{s_2s_3}=\frac{as_3m_2+as_2m_1}{s_1s_2s_3}=\frac{am_1}{s_1s_2}+\frac{am_2}{s_1s_3}=\frac{a}{s_1}\frac{m_1}{s_2}+\frac{a}{s_1}\frac{m_2}{s_3},\\
        \left(\frac{a_1}{s_1}+\frac{a_2}{s_2}\right)\frac{m}{s_3}&=\frac{s_2a_1+s_1a_2}{s_1s_2}\frac{m}{s_3}=\frac{s_2a_1m+s_1a_2m}{s_1s_2s_3}=\frac{a_1m}{s_1s_3}+\frac{a_2m}{s_2s_3}=\frac{a_1}{s_1}\frac{m}{s_3}+\frac{a_2}{s_2}\frac{m}{s_3},\\
        \left(\frac{a_1}{s_1}\frac{a_2}{s_2}\right)\frac{m}{s_3}&=\frac{a_1a_2}{s_1s_2}\frac{m}{s_3}=\frac{(a_1a_2)m}{(s_1s_2)s_3}=\frac{a_1(a_2m)}{s_1(s_2s_3)}=\frac{a_1}{s_1}\frac{a_2m}{s_2s_3}=\frac{a_1}{s_1}\left(\frac{a_2}{s_2}\frac{m}{s_3}\right),\\
        \frac{1}{1}\frac{m}{s}&=\frac{1m}{1s}=\frac{m}{s}.
    \end{align*}   
\end{solution}


















