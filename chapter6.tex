\section{Chapter VI.\hspace{0.2em} Linear algebra}
\subsection{\textsection1. Free modules revisited}

\begin{problem}[1.1]
	$\neg$ Prove that $\mathbb{R}$ and $\mathbb{C}$ are isomorphic as $\mathbb{Q}$-vector spaces. (In particular, $(\mathbb{R},+$ ) and $(\mathbb{C},+)$ are isomorphic as groups.) [II.4.4]
\end{problem}




\subsection{\textsection4. Presentations and resolutions}
\begin{problem}[4.1]
$\triangleright$ Prove that if $R$ is an integral domain and $M$ is an $R$-module, then $\operatorname{Tor}(M)$ is a submodule of $M$. Give an example showing that the hypothesis that $R$ is an integral domain is necessary. $[\S 4.1]$
\end{problem}
\begin{solution}
Given any $m_1,m_2\in\operatorname{Tor}(M)$, we can suppose there exist $r_1,r_2\in R$ such that $r_1m_1=0(r_1\ne0)$ and $r_2m_2=0(r_2\ne0)$. Since $R$ is an integral domain, we have $r_1r_2\ne0$. Thus, there exist nonzero element $r_1r_2\in R$ such that 
$$
r_1r_2(m_1+m_2)=r_2(r_1m_1)+r_1(r_2m_2)=0,
$$
which implies $m_1+m_2\in\operatorname{Tor}(M)$.

Given any $r\in R$, $m\in\operatorname{Tor}(M)$, we can suppose
\[
r_0m=0
\]
where $r_0\in R$ and $r_0\ne0$. Thus we have
\[
r_0(rm)=r(r_0m)=0,
\]
which implies $rm\in\operatorname{Tor}(M)$. Therefore, we show that $\operatorname{Tor}(M)$ is a submodule of $M$.

$\mathbb{Z}/6\mathbb{Z}$ is not an integral domain and $\mathbb{Z}/6\mathbb{Z}$ is a $\mathbb{Z}/6\mathbb{Z}$-module. Since
\[
\operatorname{Tor}(\mathbb{Z}/6\mathbb{Z})=\{[0]_6,[2]_6,[3]_6\},
\]
we have $[2]_6+[3]_6=[5]_6\notin \operatorname{Tor}(\mathbb{Z}/6\mathbb{Z})$, which implies $\operatorname{Tor}(\mathbb{Z}/6\mathbb{Z})$ is not a submodule of $\mathbb{Z}/6\mathbb{Z}
++$.
\end{solution}





\begin{problem}[4.4]
$\triangleright$ Let $R$ be a commutative ring, and $M$ an $R$-module.
\begin{itemize}
	\item Prove that $\operatorname{Ann}(M)$ is an ideal of $R$.
	\item If $R$ is an integral domain and $M$ is finitely generated, prove that $M$ is torsion if and only if $\operatorname{Ann}(M) \neq 0$.
	\item Give an example of a torsion module $M$ over an integral domain, such that $\operatorname{Ann}(M)=0 .$ (Of course this example cannot be finitely generated!)
\end{itemize}
$[\S 4.2, \S 5.3]$
\end{problem}
\begin{solution}
\begin{itemize}
	\item $\operatorname{Ann}(M)$ is defined as
	\[
	\operatorname{Ann}(M)=\{r \in R \mid \forall m \in M, r m=0\}.
	\]
	Given any $r_1,r_2\in\operatorname{Ann}(M)$, we have 
	\[
	\forall m\in M, (r_1+r_2)m=r_1m+r_2m=0,
	\]
	which implies $r_1+r_2\in\operatorname{Ann}(M)$.
	
	Given any $r\in R$, $s\in\operatorname{Ann}(M)$, we have
	\[
	\forall m\in M, (rs)m=r(sm)=0,
	\]
	which implies $rs\in\operatorname{Ann}(M)$.
	\item If $\operatorname{Ann}(M) \neq 0$, there exists nonzero $r\in R$ such that and $rm=0$ for any $m\in M$, which implies $M$ is a torsion module.
	
	Assume that $M$ is torsion and is generated by $\{m_1,m_2,\cdots,m_n\}$. There exist nonzero elements $r_1,r_2,\cdots,r_n\in R$ such that
	\[
	r_1m_1=r_2m_2=\cdots=r_nm_n=0.
	\]
	Let $r_0=r_1r_2\cdots r_n$. Since $R$ is an integral domain, $r_0\ne0$. Given any $m\in M$, we have
	\[
	m=k_1m_1+k_2m_2+\cdots+k_nm_n
	\]
	and
	\[
	r_0m=k_1r_2\cdots r_n(r_1m_1)+\cdots+k_nr_1\cdots r_{n-1} (r_nm_n)=0.
	\]
	Thus we show $r_0\in\operatorname{Ann}(M)$ and $\operatorname{Ann}(M)\ne0$.
	\item $\mathbb{Q}/\mathbb{Z}$ is a module over the integral domain $\mathbb{Z}$. Since
	\[
	\forall \left[\frac{p}{q}\right]\in \mathbb{Q}/\mathbb{Z},\ \exists q\in\mathbb{Z},\ q\left[\frac{p}{q}\right]=[0],
	\]
	we see $\mathbb{Q}/\mathbb{Z}$ is a torsion module. If $r\in\operatorname{Ann}(M)$, then for all $q\in\mathbb{Z}$,
	\[
	r\left[\frac{1}{q}\right]=0\iff q|r\implies r=0\text{ or } r\ge q.
	\]
	Therefore, there must be $r=0$, which means $\operatorname{Ann}(M)=0$.
	
\end{itemize}
\end{solution}