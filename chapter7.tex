\section{Chapter VII.\hspace{0.2em} Fields}
\subsection{\textsection1. Field extensions, I}
\begin{problem}[1.1]
$\triangleright$ Prove that if $k \subseteq K$ is a field extension, then $\mathrm{char}\,k=\mathrm{char}\,K$. Prove that the category $\mathsf{Fld}$ has no initial object. $[\S 1.1]$
\end{problem}
\begin{solution}
	Since $\mathbb{Z}$ is initial in $\mathsf{Ring}$, we have unique ring homomorphisms $i_k: \mathbb{Z} \rightarrow k$ and $i_K: \mathbb{Z} \rightarrow K$. Note that the inclusion $j:k\rightarrow K$ is a ring homomorphism, we have $i_K=j\circ i_k$. Since
	\[
	\ker i_K=\ker(j\circ i_k)=i_{k}^{-1}(\ker j)=i_{k}^{-1}(\{0\})=\ker i_k,
	\]
	we have $\mathrm{char}\,k=\mathrm{char}\,K$. If there exists one initial object $A$ in $\mathsf{Fld}$, all the objects in $\mathsf{Fld}$ will have the same characteristic as $A$. This contradicts with the fact that $\mathrm{char}\,{\mathbb{Z}}_2 \ne \mathrm{char}\,\mathbb{Z}_3$.
\end{solution}

\begin{problem}[1.3]
$\triangleright$ Let $k \subseteq F$ be a field extension, and let $\alpha \in F .$ Prove that the field $k(\alpha)$ consists of all the elements of $F$ which may be written as rational functions in $\alpha$, with coefficients in $k$. Why does this not give (in general) an onto homomorphism $k(t) \rightarrow k(\alpha) ?[\S 1.2, \S 1.3]$
\end{problem}
\begin{solution}
Since $k(\alpha)$ is smallest subfield of $F$ containing both $k$ and $\alpha$, for any $g(t)\in k(t)$ we have $g(\alpha)\in k(\alpha)$. Thus we can define the evaluation mapping
\begin{align*}
	\mathrm{ev}_{\alpha}:k(t) &\longrightarrow k(\alpha)\\
	g(t)&\longmapsto g(\alpha).
\end{align*}
It is clear to see $\mathrm{ev}_{\alpha}(k(t))\subseteq k(\alpha)$ and $k(\alpha)\subseteq \mathrm{ev}_{\alpha}(k(t))$, which means $k(\alpha)= \mathrm{ev}_{\alpha}(k(t))$.
If $\mathrm{ev}_{\alpha}:k(t) \rightarrow k(\alpha)$ is an onto field homomorphism, then $\mathrm{ev}_{\alpha}$ must be a field isomorphism. If we consider the simple extension $\mathbb{Q}\subseteq\mathbb{Q}(\sqrt{2})$, we will find that  
\[
\mathrm{ev}_{\sqrt{2}}(0)=\mathrm{ev}_{\sqrt{2}}(t^2-1)=0,
\]
which contradicts the fact that $\mathrm{ev}_{\sqrt{2}}$ is an isomorphism.
\end{solution}


\begin{problem}[1.4]
Let $k \subseteq k(\alpha)$ be a simple extension, with $\alpha$ transcendental over $k$. Let $E$ be a subfield of $k(\alpha)$ properly containing $k$. Prove that $k(\alpha)$ is a finite extension of $E$.
\end{problem}
\begin{solution}
The field extension $E\subseteq k(\alpha)$ is finitely generated because  $E(\alpha)= k(\alpha)$. Since $E$ be a subfield of $k(\alpha)$ properly containing $k$, we can suppose that there exist $f,g\in k[t]$ such that $g\ne0$, $\deg f>0$ and
\[
\frac{f(\alpha)}{g(\alpha)}\in E.
\]
Then let
\[
h(t)=f(t)-\frac{f(\alpha)}{g(\alpha)}g(t)\in E[t].
\]
It is immediate that $h(\alpha)=0$, which means $\alpha$ is algebraic over $E$. Thus we show that $E\subseteq E(\alpha)$ is a finite extension, or equivalently $k(\alpha)$ is a finite extension of $E$.
\end{solution}

\begin{problem}[1.5]
$\triangleright$ (Cf. Example 1.4.)
	\begin{itemize}
	\item Prove that there is exactly one subfield of $\mathbb{R}$ isomorphic to $\mathbb{Q}[t] /\left(t^{2}-2\right)$.
	\item  Prove that there are exactly three subfields of $\mathbb{C}$ isomorphic to $\mathbb{Q}[t] /\left(t^{3}-2\right)$.
	\end{itemize}
From a `topological' point of view, one of these copies of $\mathbb{Q}[t] /\left(t^{3}-2\right)$ looks very different from the other two: it is not dense in $\mathbb{C}$, but the others are. $[\S 1.2]$
\end{problem}
\begin{solution}
	\begin{itemize}
		\item We will show that $\mathbb{Q}(\sqrt{2})$ is the unique subfield of $\mathbb{R}$ isomorphic to $\mathbb{Q}[t] /\left(t^{2}-2\right)$. 
  
		First we assert that $\mathbb{Q}[t] /\left(t^{2}-2\right)\cong\mathbb{Q}(\sqrt{2})$. To prove this, notice that $\mathbb{Q}[t]$ is a PID and $t^2-2$ is irreducible over $\mathbb{Q}[t]$, which implies $\left(t^{2}-2\right)$ is a maximal ideal of $\mathbb{Q}[t]$ and $\mathbb{Q}[t] /\left(t^{2}-2\right)$ is a field. Then we can check that
		\begin{align*}
			\varphi:\mathbb{Q}[t] /\left(t^{2}-2\right)&\longrightarrow \mathbb{Q}(\sqrt{2}) \\
			p(t)+(t^2-2)&\longmapsto p(\sqrt{2})
		\end{align*}
		is an isomorphism.

		Suppose that $F$ is a subfield of $\mathbb{R}$ and $\psi:\mathbb{Q}[t] /\left(t^{2}-2\right)\to F$ is an isomorphism. Let $\bar{t}=t+(t^2-2)\in \mathbb{Q}[t] /\left(t^{2}-2\right)$. Since $\psi(\bar{t})\in F\subseteq\mathbb{R}$, we have
		\[
			\psi\left(\bar{t}\right)^2-2=\psi\left(\overline{t}^2-2\right)=0\implies \psi(\bar{t})=\sqrt{2}\text{ or }-\sqrt{2}\implies \mathbb{Q}(\sqrt{2})\subseteq F.
		\]
		For any $x\in F$, there exists $g=q_1t+q_2+\left(t^{2}-2\right)\in\mathbb{Q}[t] /\left(t^{2}-2\right)$ such that $\psi(g)=x$, where $q_1,q_2\in\mathbb{Q}$. If $q_1=0$, we have
		\[
			x-q_2=\psi(g-q_2)=0\implies x=q_2\in \mathbb{Q}(\sqrt{2}).
		\]
		If $q_1\ne0$, we have
		\[
			(x-q_2)^2-2q_1^2=\psi((g-q_2)^2-2q_1^2)=\psi(q_1^2t^2-2q_1^2+(t^2-2))=0,
		\]
		which implies $x=\pm q_1\sqrt{2}+q_2\in \mathbb{Q}(\sqrt{2})$. Thus we have $F\subseteq\mathbb{Q}(\sqrt{2})$. Therefore $F=\mathbb{Q}(\sqrt{2})$, which guarantees the uniqueness of $\mathbb{Q}(\sqrt{2})$.
		\item  We can show that 
		\begin{align*}
			\mathbb{Q}(\sqrt[3]{2}),\mathbb{Q}\left(\frac{-1+\sqrt{3}i}{2}\sqrt[3]{2}\right),\mathbb{Q}\left(\frac{-1-\sqrt{3}i}{2}\sqrt[3]{2}\right)
		\end{align*}
		are all isomorphic to $\mathbb{Q}[t] /\left(t^{3}-2\right)$. $\mathbb{Q}(\sqrt[3]{2})$ is not dense in $\mathbb{C}$.
	\end{itemize}
\end{solution}

\begin{problem}[1.6]
$\triangleright$ Let $k \subseteq F$ be a field extension, and let $f(x) \in k[x]$ be a polynomial. Prove that $\operatorname{Aut}_{k}(F)$ acts on the set of roots of $f(x)$ contained in $F$. Provide examples showing that this action need not be transitive or faithful. [\S1.2, \S1.3]
\end{problem}
\begin{solution}
Let $S_f$ be the set of roots of $f(x)$ contained in $F$. Note that for any $\sigma\in\operatorname{Aut}_{k}(F)$ and any $\alpha\in S_f$, $\sigma$ can be extend to $F[x]$ and $\sigma(f)=f$. Since
\[
f(\sigma(\alpha))=\sigma(f)(\sigma(\alpha))=\sigma(f(\alpha))=\sigma(0)=0\implies\sigma(\alpha)\in S_f,
\]
we can define the mapping
\begin{align*}
	\cdot:\operatorname{Aut}_{k}(F)\times S_f &\longrightarrow S_f\\
	(\sigma, \alpha) &\longmapsto  \sigma\cdot\alpha:=\sigma(\alpha).
\end{align*}
Since for any $\sigma_1,\sigma_2\in\operatorname{Aut}_{k}(F)$ and any root $\alpha$ of $f(x)$,
\[
\sigma_1\cdot(\sigma_2\cdot\alpha)=\sigma_1\cdot\sigma_2(\alpha)=\sigma_1(\sigma_2(\alpha))=(\sigma_1\circ\sigma_2)(\alpha)=(\sigma_1\circ\sigma_2)\cdot\alpha,
\]
we see that $\operatorname{Aut}_{k}(F)$ acts on the set of roots of $f(x)$ contained in $F$.
\end{solution}


\hypertarget{Exercise VII.1.7}{}
\begin{problem}[1.7]
	Let $k \subseteq F$ be a field extension, and let $\alpha \in F$ be algebraic over $k$.
	\begin{itemize}
		\item Suppose $p(x) \in k[x]$ is an irreducible monic polynomial such that $p(\alpha)=0$; prove that $p(x)$ is the minimal polynomial of $\alpha$ over $k$, in the sense of Proposition 1.3.
		\item Let $f(x) \in k[x]$. Prove that $f(\alpha)=0$ if and only if $p(x) \mid f(x)$.
		\item Show that the minimal polynomial of $\alpha$ is the minimal polynomial of a certain $k$-linear transformation of $F$, in the sense of Definition $\mathrm{VI}.6.12$.
	\end{itemize}
\end{problem}
\begin{solution}
	\begin{itemize}
		\item Suppose $q(x)$ is the minimal polynomial of $\alpha$ over $k$. Since $\deg q(x)\le \deg p(x)$, we have Euclidean division in the Euclidean domain $k[x]$ as follows
		\[
			p(x) = m(x)q(x) + r(x),\quad \deg r(x)<\deg q(x).
		\]
		Taking $x=\alpha$, there must be $r(\alpha)=0$. Since $q(x)$ is the minimal polynomial, we can assert $r(x)=0$ and accordingly $q(x)\mid p(x)$. The irreducibility of $p(x)$ means $p(x) = q(x)$.
		\item It is clear that $p(x) \mid f(x)\implies f(\alpha)=p(\alpha)h(\alpha)=0$. To show $f(\alpha)=0\implies p(x) \mid f(x)$, we can just follow the same procedure as the first problem.
		\item Suppose $p(x) \in k[x]$ is the minimal polynomial of $\alpha$. We can easily check that
		\begin{align*}
			T_\alpha:F&\longrightarrow F,\\
			 m&\longmapsto\alpha m
		\end{align*} 
		is a $k$-linear transformation. Since for any $m\in F$,
		\begin{align*}
			p(T_\alpha)(m)&=c_0+c_1T_\alpha m+c_2T_\alpha^2 m+\cdots+c_nT_\alpha^n m\\
			&=c_0+c_1\alpha m+c_2\alpha^2 m+\cdots+c_n\alpha^n m\\
			&=\left(c_0+c_1\alpha +c_2\alpha^2 +\cdots+c_n\alpha^n\right) m=0,
		\end{align*}
		we have $p(T_\alpha)=0$. Since $p(x)$ is an irreducible monic polynomial, $p(x)$ is exactly the minimal polynomial of $T_\alpha$.
	\end{itemize}
\end{solution}

\begin{problem}[1.8]
	$\neg$ Let $f(x) \in k[x]$ be a polynomial over a field $k$ of degree $d$, and let $\alpha_{1}, \ldots, \alpha_{d}$ be the roots of $f(x)$ in an extension of $k$ where the polynomial factors completely. For a subset $I \subseteq\{1, \ldots, d\}$, denote by $\alpha_{I}$ the sum $\sum_{i \in I} \alpha_{i}$. Assume that $\alpha_{I} \in k$ only for $I=\emptyset$ and $I=\{1, \ldots, d\}$. Prove that $f(x)$ is irreducible over $k$. [7.14]
\end{problem}
\begin{solution}
	Suppose $f(x)=(x-\alpha_1)\cdots(x-\alpha_d)$ is reducible over $k$ and $g(x)=\prod_{i\in I} (x-\alpha_{i})\in k[x]$ is a factor of $f$, where $I\subsetneq \{1, \ldots, d\}$. Note the coefficient of the $n-1$-th degree term of $g(x)$ is 
	\[
		-\sum_{i \in I} \alpha_{i}\notin k.
	\]
	We derive a contradiction. Hence we can conclude that $f(x)$ is irreducible over $k$.
\end{solution}

\begin{problem}[1.10]
	$\neg$ Let $k$ be a field. Prove that the ring of square $n \times n$ matrices $\mathcal{M}_{n}(k)$ contains an isomorphic copy of every extension of $k$ of degree $\leq n$. (Hint: If $k \subseteq F$ is an extension of degree $n$ and $\alpha \in F$, then `multiplication by $\alpha$' is a $k$-linear transformation of $F$.) [5.20]
\end{problem}
\begin{solution}
	Suppose $k\subseteq F$ is an extension of degree $n$ and $\alpha \in F$, then 
	\begin{align*}
		T_\alpha:F&\longrightarrow F,\\
		 m&\longmapsto\alpha m
	\end{align*} 
	is a $k$-linear transformation of $F$, which is isomorphic to a matrix $A_{T_\alpha}\in \mathcal{M}_{n}(k)$ in $\mathsf{Vect}_k$. What is left to do is to show that $A_{T_{\alpha+\beta}}=A_{T_{\alpha}}+A_{T_{\beta}}$, $A_{T_{\alpha\beta}}=A_{T_{\alpha}}A_{T_{\beta}}$ and $A_{T_{1_F}}=I_{n\times n}$, which amounts to $T_{\alpha\beta}=T_{\alpha}+T_{\beta}$, $T_{\alpha\beta}=T_{\alpha}T_{\beta}$ and $T_{1_F}=\mathrm{id}_F$. The verification is straightforward.
\end{solution}

\hypertarget{Exercise VII.1.11}{}
\begin{problem}[1.11]
	$\neg$ Let $k \subseteq F$ be a finite field extension, and let $p(x)$ be the characteristic polynomial of the $k$-linear transformation of $F$ given by multiplication by $\alpha$. Prove that $p(\alpha)=0$.

	\hspace{2em}This gives an effective way to find a polynomial satisfied by an element of an extension. Use it to find a polynomial satisfied by $\sqrt{2}+\sqrt{3}$ over $\mathbb{Q}$, and compare this method with the one used in Example 1.19. [1.12]
\end{problem}
\begin{solution}
	In \hyperlink{Exercise VII.1.7}{Exercise VII.1.7} we show that the minimal polynomial of $\alpha$ coincides with the minimal polynomial of $T_\alpha$. Denote the minimal polynomial as $f(x)$. Then we have $f(\alpha)=0$ and $f(x)\mid p(x)$, which implies $p(\alpha)=0$. 

	Consider the field extension $\mathbb{Q}\subseteq\mathbb{Q}(\sqrt{2},\sqrt{3})$ and the algebraic element $\alpha=\sqrt{2}+\sqrt{3}$. Since $1, \sqrt{2}, \sqrt{3}, \sqrt{6}$ is a basis of $\mathbb{Q}(\sqrt{2},\sqrt{3})$, we can represent the linear transformation by a matrix $A$ as follows
	\begin{align*}
		&\hspace{15pt}\left(T_\alpha 1,T_\alpha\sqrt{2},T_\alpha\sqrt{3},T_\alpha\sqrt{6}\right)\\
		&=\left(\sqrt{2}+\sqrt{3},2+\sqrt{6},3+\sqrt{6},3\sqrt{2}+2\sqrt{3}\right)\\
		&=\left(1,\sqrt{2},\sqrt{3},\sqrt{6}\right)
		\begin{pmatrix}
			0&2&3&0\\
			1&0&0&3\\
			1&0&0&2\\
			0&1&1&0
		\end{pmatrix}\\
		&=\left(1,\sqrt{2},\sqrt{3},\sqrt{6}\right)A.
	\end{align*}
	The characteristic polynomial of $A$ is
	\[
		p(x)=\det(xI-A)=x^4-10x^2+1.
	\]
\end{solution}

\hypertarget{Exercise VII.1.12}{}
\begin{problem}[1.12]
	$\neg$ Let $k \subseteq F$ be a finite field extension, and let $\alpha \in F$. The norm of $\alpha$, $N_{k \subseteq F}(\alpha)$, is the determinant of the linear transformation of $F$ given by multiplication by $\alpha$ (cf. \hyperlink{Exercise VII.1.11}{Exercise VII.1.11}, Definition VI.6.4).
	Prove that the norm is multiplicative: for $\alpha, \beta \in F$,
	$$
	N_{k \subseteq F}(\alpha \beta)=N_{k \subseteq F}(\alpha) N_{k \subseteq F}(\beta) .
	$$
	Compute the norm of a complex number viewed as an element of the extension $\mathbb{R} \subseteq \mathbb{C}$ (and marvel at the excellent choice of terminology). Do the same for elements of an extension $\mathbb{Q}(\sqrt{d})$ of $\mathbb{Q}$, where $d$ is an integer that is not a square, and compare the result with  \hyperlink{Exercise III.4.10}{Exercise III.4.10}. [1.13, 1.14, 1.15, 4.19, 6.18, VIII.1.5]
\end{problem}
\begin{solution}
	consider the field extension $\mathbb{R} \subseteq \mathbb{C}$ and a complex number $w=a+bi\in \mathbb{C}$. Given a basis $1$, $i$ of $\mathbb{C}$, denote the matrix representation of the linear transformation $F:z\mapsto wz$ as $A$. Then we have
	\begin{align*}
		&\hspace{15pt}\left(F(1),F(i)\right)\\
		&=\left(a+bi,-b+ai\right)\\
		&=\left(1,i\right)
		\begin{pmatrix}
			a&-b\\
			b&a\\
		\end{pmatrix}\\
		&=\left(1,i\right)A.
	\end{align*} 
	The norm $N_{\mathbb{R} \subseteq \mathbb{C}}(w)=\det(A)=a^2+b^2$.
	
	consider the field extension $\mathbb{Q} \subseteq \mathbb{Q}\left(\sqrt{d}\right)$ and a number $c=a+b\sqrt{d}\in \mathbb{Q}\left(\sqrt{d}\right)$. Given a basis $1$, $\sqrt{d}$ of $ \mathbb{Q}\left(\sqrt{d}\right)$, denote the matrix representation of the linear transformation $F:x\mapsto cx$ as $B$. Then we have
	\begin{align*}
		&\hspace{15pt}\left(F(1),F\left(\sqrt{d}\right)\right)\\
		&=\left(a+b\sqrt{d},bd+a\sqrt{d}\right)\\
		&=\left(1,\sqrt{d}\right)
		\begin{pmatrix}
			a&bd\\
			b&a\\
		\end{pmatrix}\\
		&=\left(1,\sqrt{d}\right)B.
	\end{align*} 
	The norm $N_{\mathbb{Q} \subseteq \mathbb{Q}(\sqrt{d})}(c)=\det(B)=a^2-db^2$.

\end{solution}

\hypertarget{Exercise VII.1.13}{}
\begin{problem}[1.13]
$\neg$ Define the trace $\operatorname{tr}_{k \subseteq F}(\alpha)$ of an element $\alpha$ of a finite extension $F$ of a field $k$ by following the lead of \hyperlink{Exercise VII.1.12}{Exercise VII.1.12}. Prove that the trace is additive:
$$
\operatorname{tr}_{k \subseteq F}(\alpha+\beta)=\operatorname{tr}_{k \subseteq F}(\alpha)+\operatorname{tr}_{k \subseteq F}(\beta)
$$
for $\alpha, \beta \in F$. Compute the trace of an element of an extension $\mathbb{Q} \subseteq \mathbb{Q}(\sqrt{d})$, for $d$ an integer that is not a square. [1.14, 1.15, 4.19, VIII.1.5]
\end{problem}
\begin{solution}
	Given $c=a+b\sqrt{d}\in \mathbb{Q}\left(\sqrt{d}\right)$, the trace $\mathrm{tr}_{\mathbb{Q} \subseteq \mathbb{Q}(\sqrt{d})}(c)=\operatorname{tr}(B)=2a$.
\end{solution}

\begin{problem}[1.14]
$\neg$ Let $k \subseteq k(\alpha)$ be a simple algebraic extension, and let $x^{d}+a_{d-1} x^{d-1}+\cdots+a_{0}$ be the minimal polynomial of $\alpha$ over $k$. Prove that
$$
\operatorname{tr}_{k \subseteq k(\alpha)}(\alpha)=-a_{d-1} \quad \text { and } \quad N_{k \subseteq k(\alpha)}(\alpha)=(-1)^{d} a_{0} .
$$
(Cf. \hyperlink{Exercise VII.1.12}{Exercise VII.1.12} and \hyperlink{Exercise VII.1.13}{Exercise VII.1.13}). [4.19]
\end{problem}
\begin{solution}
	Since $p(x)=x^{d}+a_{d-1} x^{d-1}+\cdots+a_{0}$ is the minimal polynomial of $\alpha$, we see $1,\alpha,\cdots,\alpha^{d-1}$ is the basis of the $k$-vector space $k(\alpha)$ and  
	\begin{align*}
		p(\alpha)=\alpha^{d}+a_{d-1} \alpha^{d-1}+\cdots+a_{0}=0
	\end{align*}	
	The matrix representation of the linear map $F:z\mapsto \alpha z$ is
	\begin{align*}
		&\hspace{15pt}\left(F(1),F\left(\alpha\right),\cdots,F\left(\alpha^{d-1}\right)\right)\\
		&=\left(1,\alpha,\alpha^2,\cdots,\alpha^{d-1}\right)
		\begin{pmatrix}
			0&0&0&\cdots&0&-a_0\\
			1&0&0&\cdots&0&-a_1\\
			0&1&0&\cdots&0&-a_2\\
			\vdots&\vdots&\vdots&&\vdots&\vdots\\
			0&0&0&\cdots&0&-a_{d-2}\\
			0&0&0&\cdots&1&-a_{d-1}
		\end{pmatrix}\\
		&=\left(1,\alpha\right)A. 	
	\end{align*}
Thus we have
	\begin{align*}
		\operatorname{tr}_{k \subseteq k(\alpha)}(\alpha)=\operatorname{tr}(A)=-a_{d-1}
	\end{align*}
	and
	\begin{align*}
		 N_{k \subseteq k(\alpha)}(\alpha)=\det(A)=(-1)^{d-1} (-a_{0})= (-1)^{d} a_{0}.
	\end{align*}
\end{solution}


\begin{problem}[1.16]
$\triangleright$ Let $k \subseteq L \subseteq F$ be fields, and let $\alpha \in F$. If $k \subseteq k(\alpha)$ is a finite extension, then $L \subseteq L(\alpha)$ is finite and $[L(\alpha): L] \leq[k(\alpha): k]$. [§1.3]
\end{problem}
\begin{solution}
	Since $k \subseteq k(\alpha)$ is a finite extension, it is a simple algebraic extension and there exists a minimal polynomial $p(t)$ of $\alpha$ over $k$. Since $p(t)$ can be seen as a polynomial in $L$ and $p(\alpha)=0$, we have $L \subseteq L(\alpha)$ is finite. Since the degree of the minimal polynomial of $\alpha$ over $L$ is not greater than the degree of $p(t)$, we have $[L(\alpha): L] \leq[k(\alpha): k]$.
\end{solution}

\begin{problem}[1.20]
Let $p$ be a prime integer, and let $\alpha=\sqrt[p]{2} \in \mathbb{R}$. Let $g(x) \in \mathbb{Q}[x]$ be any nonconstant polynomial of degree $<p$. Prove that $\alpha$ may be expressed as a polynomial in $g(\alpha)$ with rational coefficients.\\
Prove that an analogous statement for $\sqrt[4]{2}$ is false.
\end{problem}
\begin{solution}
It is easy to see that $\mathbb{Q}\subseteq \mathbb{Q}(g(\alpha)) \subseteq\mathbb{Q}(\alpha)$. Since $f(t)=x^p-2$ is irredicible over $\mathbb{Q}$ and $f(\alpha)=0$, $\mathbb{Q}\subseteq \mathbb{Q}(\alpha)$ is a $p$ degree extension. If $p$ is a prime integer, then the degree fo the finite extension $\mathbb{Q}\subseteq \mathbb{Q}(g(\alpha))$ is 1 or $p$. Note that $g(\alpha)\notin \mathbb{Q}$. There must be $\mathbb{Q}(g(\alpha))=\mathbb{Q}(\alpha)$. Thus we see there exists $h\in \mathbb{Q}[x]$ such that $\alpha = h(g(\alpha))$ with rational coefficients.

If $p=4$, then we can take $g(x)=x^2$. Now we have $g(\alpha)=\sqrt{2}$. For the tower of field extensions $\mathbb{Q}\subseteq \mathbb{Q}(\sqrt{2}) \subseteq\mathbb{Q}(\sqrt[4]{2})$, it is easy to check that both $\mathbb{Q}\subseteq \mathbb{Q}(\sqrt{2}) $ and $ \mathbb{Q}(\sqrt{2}) \subseteq\mathbb{Q}(\sqrt[4]{2})$ are quadratic extensions. Hence the intermediate field $ \mathbb{Q}(\sqrt{2})$ is not equal to $\mathbb{Q}$ or $\mathbb{Q}(\sqrt[4]{2})$. Therefore, we can conclude that $\alpha=\sqrt[4]{2}$ cannot be expressed as a polynomial in $g(\alpha)=\sqrt{2}$ with rational coefficients.
\end{solution}


\begin{problem}[1.22]
	 Let $k \subseteq F$ be a field extension, and let $\alpha \in F, \beta \in F$ be algebraic, of degree $d, e$, resp. Assume $d, e$ are relatively prime, and let $p(x)$ be the minimal polynomial of $\beta$ over $k$. Prove $p(x)$ is irreducible over $k(\alpha)$.
	\end{problem}
\begin{solution}
	By the tower properties of finite field extensions $k \subseteq k(\alpha) \subseteq k(\alpha,\beta)$ and  $k \subseteq k(\beta) \subseteq k(\alpha,\beta)$, we have $d$ divides $[k(\alpha,\beta): k]$ and $e$ divides $[k(\alpha,\beta): k]$. Since $d$ and $e$ are relatively prime, we have $[k(\alpha,\beta): k]\ge de$. Since $p(\beta)=0$, the minimal polynomial of $\beta$ over $k(\alpha)$ divides $p(x)$, which implies $ [k(\alpha)(\beta): k(\alpha)]\le \deg(p)=e$. By the tower property of finite field extensions $k \subseteq k(\alpha) \subseteq k(\alpha,\beta)$, we have $[k(\alpha,\beta): k]\le de$, which forces $[k(\alpha,\beta): k]= de$ and $[k(\alpha)(\beta): k(\alpha)]=\deg(p)=e$. Thus we see $p(x)$ is the minimal polynomial of $\beta$ over $k(\alpha)$, and hence $p(x)$ is irreducible over $k(\alpha)$.
\end{solution}

\begin{problem}[1.23]
Express $\sqrt{2}$ explicitly as a polynomial function in $\sqrt{2}+\sqrt{3}$ with rational coefficients.
\end{problem}
\begin{solution}
	Suppose $f(x)=a+b x+c x^2+d x^3$ is a polynomial in $\mathbb{Q}[x]$.
	\begin{align*}
		&\;a+b\left(\sqrt{2}+\sqrt{3}\right)+c\left(\sqrt{2}+\sqrt{3}\right)^2+d\left(\sqrt{2}+\sqrt{3}\right)^3\\
		=&\;a+b\left(\sqrt{2}+\sqrt{3}\right)+c\left(5+2\sqrt{6}\right)+d\left(11\sqrt{2}+9\sqrt{3}\right)\\
		=&\;a+5c+(b+11d)\sqrt{2}+(b+9d)\sqrt{3}+2c \sqrt{6}
	\end{align*}
	Solve
	\begin{align*}
		a+5c&=0\\
		b+11d&=1\\
		b+9d&=0\\
		2c&=0
	\end{align*}
	and we get $a=0, b=-\frac{9}{2}, c=0, d=\frac{1}{2}$. Therefore, $\sqrt{2}$ can be expressed as
	\[
		\sqrt{2}=f\left(\sqrt{2}+\sqrt{3}\right)=\frac{1}{2}\left(\sqrt{2}+\sqrt{3}\right)^3-\frac{9}{2}\left(\sqrt{2}+\sqrt{3}\right)	.
	\]
\end{solution}

\begin{problem}[1.25]
$\neg$ Let $\xi:=\sqrt{2+\sqrt{2}}$.
	\begin{itemize}
		\item Find the minimal polynomial of $\xi$ over $\mathbb{Q}$, and show that $\mathbb{Q}(\xi)$ has degree 4 over $\mathbb{Q}$.
		\item Prove that $\sqrt{2-\sqrt{2}}$ is another root of the minimal polynomial of $\xi$.
		\item Prove that $\sqrt{2-\sqrt{2}} \in \mathbb{Q}(\xi)$. (Hint: $\left.(a+b)(a-b)=a^2-b^2.\right)$
		\item By Proposition 1.5, sending $\xi$ to $\sqrt{2-\sqrt{2}}$ defines an automorphism of $\mathbb{Q}(\xi)$ over $\mathbb{Q}$. Find the matrix of this automorphism w.r.t. the basis $1, \xi, \xi^2, \xi^3$.
		\item Prove that $\operatorname{Aut}_{\mathbb{Q}}(\mathbb{Q}(\xi))$ is cyclic of order 4.
	\end{itemize}
[6.6]
\end{problem}
\begin{solution}
	\begin{itemize}
		\item Let 
		\[
			p(x)=\left(x^2-2\right)^2-2=x^4-4x^2+2.
		\]
		By Eisenstein's criterion, $p(x)$ is irreducible over $\mathbb{Q}$. Since $p(\xi)=0$, $p(x)$ is the minimal polynomial of $\xi$ over $\mathbb{Q}$. Therefore, $\mathbb{Q}(\xi)$ has degree 4 over $\mathbb{Q}$.
		\item It is straightforward to check that $p\left(\sqrt{2-\sqrt{2}}\right)=0$.
		\item 
		\begin{align*}
			\sqrt{2-\sqrt{2}}=\sqrt{\frac{\left(2-\sqrt{2}\right)\left(2+\sqrt{2}\right)}{2+\sqrt{2}}}=\frac{\sqrt{2}}{\sqrt{2+\sqrt{2}}}=\frac{\xi^2-2}{\xi}
		\end{align*}
		\item Suppose $\sigma$ is an automorphism of $\mathbb{Q}(\xi)$ over $\mathbb{Q}$ sending $\xi$ to $\sqrt{2-\sqrt{2}}$. We need express $\sigma(1),\sigma(\xi),\sigma(\xi^2),\sigma(\xi^3)$ as the linear combination of $1, \xi, \xi^2, \xi^3$. Note 
		\begin{align*}
			\sigma(\xi)&=\sqrt{2-\sqrt{2}}=\xi-\frac{2}{\xi}.
		\end{align*}
		We may express $\frac{1}{\xi}$ as a linear combination of $\xi$ and $\xi^3$. Suppose there exists $a,b\in \mathbb{Q}$ such that
		\[
			\frac{1}{\xi}=a\xi+b\xi^3.
		\]
		Then we have
		\begin{align*}
			&\;a\xi^2+b\xi^4\\
			=&\;a(2+\sqrt{2})+b(6+4\sqrt{2})\\
			=&\;2a+6b+(a+4b)\sqrt{2}\\
			=&\;1,
		\end{align*}
		which implies $a=2$, $b=-\frac{1}{2}$ and $\frac{1}{\xi}=2\xi-\frac{1}{2}\xi^3$. Therefore, 
		\begin{align*}
			\sigma(\xi)&=\xi-\frac{2}{\xi}=\xi-2\left(2\xi-\frac{1}{2}\xi^3\right)=-3\xi+\xi^3\\
			\sigma\left(\xi^2\right)&=\sigma(\xi)^2=2-\sqrt{2}=4-\xi^2\\
			\sigma\left(\xi^3\right)&=\left(\xi-\frac{2}{\xi}\right)\left(4-\xi^2\right)=-\xi^3+6 \xi-\frac{8}{\xi}=-\xi^3+6 \xi-8\left(2\xi-\frac{1}{2}\xi^3\right)=-10\xi+3\xi^3.
		\end{align*}
		and the matrix representation of $\sigma$ w.r.t. the basis $1, \xi, \xi^2, \xi^3$ is 
		\begin{align*}
			&\hspace{15pt}\left(\sigma(1),\sigma(\xi),\sigma(\xi^2),\sigma(\xi^3)\right)\\
			&=\left(1, \xi, \xi^2, \xi^3\right)
			\begin{pmatrix}
				1&0&4&0\\
				0&-3&0&-10\\
				0&0&-1&0\\
				0&1&0&3
			\end{pmatrix}\\
			&=\left(1, \xi, \xi^2, \xi^3\right)A.
		\end{align*}
		\item Since
		\[
			x_1=\sqrt{2+\sqrt{2}},\quad x_2=\sqrt{2-\sqrt{2}},\quad x_3=-\sqrt{2+\sqrt{2}},\quad x_4=-\sqrt{2-\sqrt{2}}
		\]
		are the all roots of $p(x)$ over $\overline{\mathbb{Q}}$ and $\mathbb{Q}(\xi)=\mathbb{Q}(x_1,x_2,x_3,x_4)$, we see $\mathbb{Q}\subseteq\mathbb{Q}(\xi)$ is a Galois extension and the Galois group $\operatorname{Aut}_{\mathbb{Q}}(\mathbb{Q}(\xi))$ is of order 4. By calculating the characteristic polynomial of $A$
		\begin{align*}
			\det\left(\lambda I-A\right)&=\det \begin{pmatrix}
				\lambda-1&0&-4&0\\
				0&\lambda+3&0&10\\
				0&0&\lambda+1&0\\
				0&-1&0&\lambda-3
			\end{pmatrix}\\
			&=(\lambda-1)\left((\lambda+3)(\lambda+1)(\lambda-3)+10(\lambda+1)\right)\\
			&=(\lambda-1)(\lambda+1)\left(\lambda^2+1\right)\\
			&=\lambda^4-1,
		\end{align*}
		we have $A^4=I$. Since $A^2\ne I$, the order of $\sigma\in \operatorname{Aut}_{\mathbb{Q}}(\mathbb{Q}(\xi))$ is 4. Therefore, $\operatorname{Aut}_{\mathbb{Q}}(\mathbb{Q}(\xi))$ is a cyclic group of order 4.
	\end{itemize}
\end{solution}

