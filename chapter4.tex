\section{Chapter IV.\hspace{0.2em} Groups, second encounter}
\subsection{\textsection1. The conjugation action}
\begin{problem}[1.1]
	$\triangleright$ Let $p$ be a prime integer, let $G$ be a $p$-group, and let $S$ be a set such that $|S| \not \equiv 0 \bmod p .$ If $G$ acts on $S,$ prove that the action must have fixed points. $[\S 1.1$ $\S 2.3]$
\end{problem}
\begin{solution}
	let $Z$ be the fixed-point set of the action. We have
	\[
	|Z| \equiv|S| \quad \bmod p.
	\]
	Since $|S| \not \equiv 0 \bmod p$, there must be $|Z|\ne0$, which implies the action must have fixed points.
\end{solution}

\begin{problem}[1.2]
Find the center of $D_{2 n}$. (The answer depends on the parity of $n$. You have actually done this already: \hyperlink{Exercise II.2.7}{Exercise II.2.7}. This time, use a presentation.)
\end{problem}
\begin{solution}
\[
D_{2n}=\left\langle r, s \mid r^{n}=s^{2}=(sr)^{2}=e\right\rangle
\]
It is clear that $r^is^j\in Z(D_{2n})$ if and only if $r^is^j$ commutes with $r$ and $s$. That is,
\[
(r^is^j)r=r(r^is^j)\iff s^jr=rs^j\iff j=0
\]
and
\[
(r^is^j)s=s(r^is^j)\iff r^is=sr^i\iff i=0\text{  or  }n.
\]
Therefore, 
\[
Z(D_{2n})=\{e,r^n\}.
\]
\end{solution}

\begin{problem}[1.4]
$\triangleright$ Let $G$ be a group, and let $N$ be a subgroup of $Z(G)$. Prove that $N$ is normal in $G .[\S 2.2]$
\end{problem}
\begin{solution}
Since for all $g\in G$, $a\in N$,
\[
gag^{-1}=agg^{-1}=a\in N,
\]
we see $N$ is normal in $G$.

\end{solution}


\begin{problem}[1.5]
$\triangleright$ Let $G$ be a group. Prove that $G / Z(G)$ is isomorphic to the group $\operatorname{Inn}(G)$ of inner automorphisms of $G .$ (Cf. \hyperlink{Exercise II.4.8}{Exercise II.4.8}.) Then prove Lemma 1.5 again by using the result of \hyperlink{Exercise II.6.7}{Exercise II.6.7}. $[\S 1.2]$
\end{problem}
\begin{solution}
Define
\begin{align*}
	f:G/Z(G)&\longrightarrow \operatorname{Inn}(G)\\
	gZ(G)&\longmapsto (\gamma_g:a\longmapsto gag^{-1}).
\end{align*}
We can check that $f$ is a homomorphism
\[
f(g_1g_2Z(G))=\gamma_{g_1g_2}=\gamma_{g_1}\gamma_{g_2}=f(g_1Z(G))f(g_2Z(G)).
\]
Since
\[
gZ(G)\in\ker f\iff \gamma_g=\mathrm{id}_G\iff\forall a\in G, gag^{-1}=a\iff g\in Z(G),
\]
we have
\[
\ker f=Z(G),
\]
which means $f$ is injective. It is clear that $f$ is surjective. Thus we show $f$ is an isomorphism and $G / Z(G)\cong\operatorname{Inn}(G)$.


\end{solution}

\begin{problem}[1.6]
$\triangleright$ Let $p, q$ be prime integers, and let $G$ be a group of order $p q .$ Prove that either $G$ is commutative, or the center of $G$ is trivial. Conclude (using Corollary 1.9$)$ that every group of order $p^{2}$, for a prime $p$, is commutative. $[\S 1.3]$
\end{problem}
\begin{solution}
Since $Z(G)\operatorname{\triangleleft} G$, it follows that  $|Z(G)|\in\{1,p,q,pq\}$. Suppose $|Z(G)|=p\text{  or  }q$, we have $|G/Z(G)|=p\text{  or  }q$. Since groups of prime orders are cyclic, $G/Z(G)$ is cyclic. According to Lemma 1.5, it implies $G / Z(G)=\{e_G\}$, which yields a contradiction. Hence $|Z(G)|=1\text{  or  }pq$, that is, 
\[
Z(G)=\{e_G\}\text{  or  }G.
\] 
Therefore, we prove that either $G$ is commutative, or the center of $G$ is trivial.

Let $G$ be a group of order $p^{2}$. According to Corollary 1.9, the center of the nontrivial $p$-group $G$ is nontrivial. Therefore, $G$ must be commutative.

\end{solution}